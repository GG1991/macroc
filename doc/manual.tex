\documentclass[conference, onecolumn]{IEEEtran}

% \usepackage[pdftex]{graphicx}
% \usepackage[dvips]{graphicx}
%\usepackage{amsmath}
\usepackage{algorithm}
\usepackage{algorithmic}

%\usepackage{array}
%\usepackage[caption=false,font=normalsize,labelfont=sf,textfont=sf]{subfig}
%\usepackage{fixltx2e}
%\usepackage{stfloats}
%\usepackage{url}

\usepackage{amsmath}
\usepackage{amssymb}
\RequirePackage{amsfonts}
\RequirePackage{standalone}
\RequirePackage{tikz}
\usetikzlibrary{matrix,backgrounds,calc,shapes,arrows,arrows.meta,fit,positioning}                                                                
\usetikzlibrary{chains,shapes.multipart}                                                                                                          
\usetikzlibrary{shapes,calc} 

\usepackage{listings}
\usepackage{color}

\definecolor{dkgreen}{rgb}{0,0.6,0}
\definecolor{gray}{rgb}{0.5,0.5,0.5}
\definecolor{mauve}{rgb}{0.58,0,0.82}

\lstset{frame=tb,
language=C,
aboveskip=3mm,
belowskip=3mm,
showstringspaces=false,
basicstyle={\small\ttfamily},
numbers=none,
numberstyle=\tiny\color{gray},
keywordstyle=\color{blue},
commentstyle=\color{dkgreen},
stringstyle=\color{mauve},
breaklines=true,
breakatwhitespace=true,
tabsize=4
}

%columns=flexible,
\hyphenation{op-tical net-works semi-conduc-tor}


\begin{document}

\title{MacroC: Reference Manual}

\author{\IEEEauthorblockN{
	Guido Giuntoli\IEEEauthorrefmark{1}\IEEEauthorrefmark{2},
	}
\IEEEauthorblockA{\IEEEauthorrefmark{1}Barcelona Supercomputing Center}
\IEEEauthorblockA{\IEEEauthorrefmark{2}guido.giuntoli@bsc.es}
}

\maketitle

% no keywords

\IEEEpeerreviewmaketitle

\section{Introduction}
\IEEEPARstart
MacroC is a structure-grid parallel finite element code used to solve the solid mechanics equations specialized for the resolution of the macro-scale problem of FE2 multi-scale calculation. This code was designed for being coupled with PETSc library, which provides the structure grid management and solver routines in parallel and MicroPP, which is a sequential code used to perform the localization-homogenization process on heterogeneous microstructures.

\hfill August 26, 2015

%\input{governing_equations_and_fe.tex}

\section{Implementation}

\section*{Acknowledgment}

%\begin{thebibliography}{1}
%\end{thebibliography}

\end{document}
